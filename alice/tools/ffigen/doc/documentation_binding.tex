\documentclass[a4paper,twoside]{article}
\NeedsTeXFormat{LaTeX2e}

\usepackage{amstext,amsmath,amssymb,amsthm,epsfig,german,}

\begin{document}

\begin{titlepage}
    \centering
    \vspace{3cm}
    \textbf{\Huge Alice - C - Binding Generator \\ }
    \vspace{3cm}
        \textbf{\Huge Dokumentation \\ }
    \vspace{3cm}
    \textbf{\huge Sven Woop\\}

    \vspace{\fill}
    {\scriptsize
    \today \\
    set with \LaTeXe \\
    \vspace{5mm}
    \textbf{\textit{email:}}\\\vspace{0.5ex}
    \textbf{\textit{woop@ps.uni-sb.de}}\\}
\end{titlepage}

\tableofcontents
\vspace{\fill}

%%%%%%%%%%%%%%%%%%%%%%%%%%%%%%%%%%%%%%%%%%%%%%%%%%%%%%%%%%%%%%%%%%%%%%%
%%%%%%%%%%%%%%%%%%%%%%%%%%%%%%%%%%%%%%%%%%%%%%%%%%%%%%%%%%%%%%%%%%%%%%%
\newpage
\section{Allgemein}
Der hier beschriebene Binding-Generator ist ein eigenst"andiges Tool mittels 
dessen C-Bibliotheken unter Alice benutzt werden k"onnen. Alice ist eine am 
Lehrstuhl von Prof. Smolka an der Universit"at Saarbr"ucken entwickelte 
Programmiersprache. Es ist anzumerken, dass das Tool nur mit C-Bibliotheken, 
nicht aber C++ Bibliotheken umgehen kann.

%%%%%%%%%%%%%%%%%%%%%%%%%%%%%%%%%%%%%%%%%%%%%%%%%%%%%%%%%%%%%%%%%%%%%%%
%%%%%%%%%%%%%%%%%%%%%%%%%%%%%%%%%%%%%%%%%%%%%%%%%%%%%%%%%%%%%%%%%%%%%%%
\section{Kommandozeilensyntax}

Das Tool wird folgenderma"sen von der Kommandozeile aufgerufen:

\begin{verbatim}
sml @SMLload=alice-c-binding -c config-file.xml -h header-file.h 
                             -i include1.h -i include2.h ...
\end{verbatim}

Es ist zu beachten, dass die virtuelle Maschine von SML-New-Jersey ben"otigt wird um das Tool zu starten. Es folgt eine kurze Beschreibung der Parameter:\\
\\
\begin{tabular}{r|p{9cm}}
 Parameter &  Beschreibung \\ \hline
 -c file  &   Gibt die zu verwendende Konfigurationsdatei an. \\ 
 -h file  &   Gibt die Quelldatei an, aus welcher Funktionen gebunden werden sollen. In dem angegebenen Headerfile m"ussen alle zu bindenden Funktionen enthalten sein. \\
 -i file  &   Mit der Option -i k"onnen weitere Dateien angegeben werden, die in das sp"atere Binding inkludiert werden. So ist es m"oglich eigene Makros zu definieren und diese in dem Binding zu verwenden. N"aheres dazu sp"ater. 
\end{tabular}\\
\\
\\
Das Binding ben"otigt immer eine Datei namens types.aml, in der Basis-C-Typen definiert sind (c\_char, c\_short, ..., 'a pointer) . Diese Datei wird mit dem folgenden Aufruf erstellt.
\\
 \begin{verbatim}
sml @SMLload=alice-c-binding --create-types
\end{verbatim}


%%%%%%%%%%%%%%%%%%%%%%%%%%%%%%%%%%%%%%%%%%%%%%%%%%%%%%%%%%%%%%%%%%%%%%%
%%%%%%%%%%%%%%%%%%%%%%%%%%%%%%%%%%%%%%%%%%%%%%%%%%%%%%%%%%%%%%%%%%%%%%%
\section{Konfigurationsdatei}

%%%%%%%%%%%%%%%%%%%%%%%%%%%%%%%%%%%%%%%%%%%%%%%%%%%%%%%%%%%%%%%%%%%%%%
\subsection{Filter}

Der Generator erstellt nur von den Funktionen und Typen ein Binding, dessen Name auf einen 
der in der Konfigurationsdatei angegebenen Filter passt. Filter k"onnen auch verwendet
werden um die Namen von Funktionen und Typen zu manipulieren. Hier ein kleiner Beispielfilter
an dem die Funktionsweise im Folgenden erl"autert wird. Wenn im folgenden im Zusammenhang mit
Filtern von Funktionen die Rede ist, so sind auch immer Typen gemeint, da sich die Filter
auf diesen genauso verhalten (abgesehen von Standard-C-Typen wie char, int, float, ...).

\begin{verbatim}
<filter>

  <accept>cfun1</accept>
  <accept>gtk_##</accept>

  <ignore>mem##</ignore>
  
  <rename>
    <from>_#x#</from>
    <to>#x#</to>
  </rename>
  
</filter>
\end{verbatim}

Dieser Filter erstellt ein Binding von der Funktion cfun1 und allen Funktionen
die mit gtk\_ beginnen. Von den Funktionen deren Name mit mem beginnt wird kein Binding
erstellt. Der Rename Block verwirft einen Unterstrich am Namensanfang einer Funktion 
und akzeptiert diese. So wird zum Beispiel die Funktion \_hello\_world zu
hello\_world in dem Binding.\\
Die Filtereintr"age (accept, ignore, rename) in einem Filter werden immer von oben 
nach unten abgearbeitet. Passt ein solcher Filtereintrag auf einen Funktionsnamen, so
wird der Filter beendet. Eine Zeile wie 

\begin{verbatim}<ignore>gtk_new</ignore>\end{verbatim} 

am Ende des obigen Filterabschnitts, h"atte also keinen Effekt, da die Funktion gtk\_new von einem 
vorherigen Filtereintrag akzeptiert wurde.\
Es k"onnen jedoch mehrere Filter hintereinandergeschaltet werden um den Effekt zu erzielen:

\begin{verbatim}
<filter>
  <accept>gtk_##</accept>
</filter>
<filter>
  <ignore>gtk_new</ignore>
  <accept>##</accept>
</filter>
\end{verbatim}

Beachten Sie, dass hier zwei Filter angegeben sind. Der erste Filter akzeptiert
alle Funktionen die mit gtk\_ beginnen. Die Funktionen die diesen Filter passiert
haben werden in den zweiten gesteckt. Dieser ignoriert die Funktion gtk\_new und l"a"st
alle anderen Funktionen ungehindert passieren. Der letzte Filtereintrag im zweiten Filter ist wichtig, da ansonsten \emph{keine} Funktion die beiden Filter passieren kann. \\
\\
Es k"onnen vielf"altige Manipulationen der Funktionsnamen vorgenommen werden. Hier ein weiteres n"utzliches Beispiel:  

\begin{verbatim}
<filter>
  <rename>
    <from>#x#_#y#_#z#</from>
    <to>#x##CAPITAL:y##CAPITAL:z#</to>
  </rename>
</filter>
\end{verbatim}

Dieser Filter wandelt Funktionen wie get\_box\_min in getBoxMin um.

%%%%%%%%%%%%%%%%%%%%%%%%%%%%%%%%%%%%%%%%%%%%%%%%%%%%%%%%%%%%%%%%%%%%%%
\subsection{Spezielle Typen}

Das Binding erlaubt es spezielle Konvertierungsmakros f"ur beliebige C-Typen 
anzugeben. So ist es zum Beispiel m"oglich einer Funktion wie 
\begin{verbatim}int sum_list(List lst);\end{verbatim}
den Typ int list $\rightarrow$ int zu geben. Es wird nur eine C-Funktion oder Makro ben"otigt welches den C-Typ
List korrekt in ein Word umwandelt, das die Liste vom Typ int list auf der Alice Seite repr"asentiert.
Diese C-Funktion nennen wir im folgenden 
\begin{verbatim}word LIST_TO_WORD(List lst);\end{verbatim} 
Die Konvertierung in die entgegengesetzte Richtung muss von einem Makro "ubernommen werden. 
Wir nennen dieses Makro 
\begin{verbatim}DECLARE_LIST(y0,x0);\end{verbatim}
welches das Argument x0 vom Typ word
zu konvertieren hat in eine Liste und das Ergebnis in der Variablen y0 zur"uckzuliefern hat.\\
\\
Sind soche Makros/Funktionen gegeben, so kann f"ur den Typ List ein Typeintrag erstellt werden. 
Dieser sieht in der Konfigurationsdatei wie folgt aus:

\begin{verbatim}
<type>
  <ctype>List</ctype>
  <alicetype>int list</alicetype>
  <toword>LIST_TO_WORD</toword>
  <fromword>DECLARE_LIST</fromword>
</type>
\end{verbatim}

Es ist zu beachten, dass auf der Kommandozeile anzugeben ist wo die Makros/Funktionen
LIST\_TO\_WORD und DECLARE\_LIST deklariert sind. Dies ist mit der Option -i m"oglich.\\
\\
Weitere Informationen zur Erstellung solcher Makros entnehmen Sie bitte der Alice-Dokumentation.

%%%%%%%%%%%%%%%%%%%%%%%%%%%%%%%%%%%%%%%%%%%%%%%%%%%%%%%%%%%%%%%%%%%%%%
\subsection{Benutzerdefinierte Funktionen}

Manchmal ist es n"otig bestimmte Funktionen selbst zu binden, oder gar eigene
Funktionen dem Binding zuzuf"ugen. Dies geschieht mit einem fun-Block. 
Hier ein kleines Beispiel von einer Identit"atsfunktion

\begin{verbatim}
<fun>
   <name>identity</name>
   <type>'a -> 'a</type>
   <export>
     INIT_STRUCTURE(record, "util", "identity", identity, 1);
   </export>
   <body>
       DEFINE1(identity) {
         RETURN(x0);
       } END
   </body>
</fun>
\end{verbatim}

Der Name-Block gibt den Namen der Funktion in Alice an und der Type-Block den Typ 
der Funktion in Alice. Die Exportzeile f"ugt die Funktion zu der Struktur util hinzu 
und der Body-Block stellt den Rumpf der Funktion dar. Da das vom den Tool erstellte Binding automatisch den Namen des angegebenen Headerfile tr"agt (siehe -h Kommandozeilenoption) muss der zweite Parameter des INIT\_STRUCTURE-Aufrufs in der Export-Zeile immer gleich dem Namen des Headerfiles sein (ohne Endung). Der dritte Parameter dieser Funktion entpricht dem Namen der Funktion in dem Binding und der vierte muss mit dem im Rumpf angegebenen Funktionsnamen "ubereinstimmen. Der letzte Parameter gibt die Anzahl der Argumente der Funktion an, und der erste muss \emph{immer} record lauten.\\
\\
Wenn man eine Funktion die in dem Headerfile angegeben ist selbst binden will, ist zu beachten,
mit einem Filtereintrag zun"achst die automatische Bindung der Funktion zu unterdr"ucken.

%%%%%%%%%%%%%%%%%%%%%%%%%%%%%%%%%%%%%%%%%%%%%%%%%%%%%%%%%%%%%%%%%%%%%%%
%%%%%%%%%%%%%%%%%%%%%%%%%%%%%%%%%%%%%%%%%%%%%%%%%%%%%%%%%%%%%%%%%%%%%%%
\section{Abbildung von C Typen auf Alice Typen}

Typen werden nach den folgenden Regeln von C-Typen in Alicetypen umgewandelt 
(von oben nach unten zu lesen):\\
\\
\begin{tabular}{r|l}
C-Typ & Alice-Typ \\ \hline
char* & c\_char pointer \\
short*& c\_short pointer \\
int*  & c\_int pointer \\
long* & c\_long pointer \\
char  & char \\
short & int \\
int   & int \\
long  & int \\
void* & 'a pointer \\
void  & unit \\
int[n] & int Array.array ... \\
struct name & type name (der abstrakte Typ name, wird erstellt) \\
union  name & type name (der abstrakte Typ name, wird erstellt) \\
\end{tabular}\\
\\
\\
Es ist anzumerken, dass f"ur Strukturen und Unions eigene Typen in Alice
definiert werden. Demnach ist darauf zu achten, dass die Namen dieser Strukturen
auch g"ultigen Namen auf der Alice Seite entsprechen. In C ist zum Beispiel ein
Typ mit dem Namen \_type zugelassen, in Alice verursacht der f"uhrende Unterstrich einen
Fehler, welcher aber durch die Namensmanipulation in den Filtern behoben werden kann.\\

Zeiger auf C-Funktionen werden in Alice zu dem Typ c\_fun. Will man eine C-Funktion A einer anderen Funktion B als
Argument "ubergeben, so muss man zuerst eine benutzerdefinierte Funktion getAPtr erstellen, die einen Zeiger auf die Funktion A zur"uckliefert. Dieser hat unter Alice den Typ c\_fun und kann als Parameter f"ur die Funktion B in Alice verwendet werden.\\

F"ur enum-Typen wird ein datatype erstellt. So wird aus 
\begin{verbatim}enum myenum { A, B, C };\end{verbatim}
der Alice-Typ 
\begin{verbatim}datatype myenum = A | B | C\end{verbatim}

%%%%%%%%%%%%%%%%%%%%%%%%%%%%%%%%%%%%%%%%%%%%%%%%%%%%%%%%%%%%%%%%%%%%%%%
%%%%%%%%%%%%%%%%%%%%%%%%%%%%%%%%%%%%%%%%%%%%%%%%%%%%%%%%%%%%%%%%%%%%%%%
\section{Behandlung von Strukturen und Unions}

Wie oben beschrieben wird f"ur Strukturen und Unions ein eigener abstrakter
Typ erstellt. Desweitern werden unter anderem f"ur alle Memberfunktionen Zugriffsfunktionen
generiert. Dazu wieder ein kurzes Beispiel:

\begin{verbatim}
struct vec {
  int x; int y;
};
\end{verbatim}

Diese C-Struktur f"uhrt zur Erstellung folgender Eintr"age in der Alice Signatur:

\begin{verbatim}
type vec
cast_vec   : 'a pointer -> vec pointer
sizeof_vec : unit -> int
vec_new    : unit -> vec pointer
vec_get_x  : unit -> int
vec_set_x  : vec pointer * int -> unit
vec_get_y  : unit -> int
vec_set_y  : vec pointer * int -> unit
\end{verbatim}

Zun"achst wird der abstrakte Typ f"ur die Struktur generiert. Die Cast-Funktion
castet einen beliebigen pointer-Typ zu einem vec pointer. Sizeof liefert die 
ben"otigten Bytes der vec-Struktur auf C-Seite zur"uck. Mit vec\_new kann Speicher
f"ur einen neuen vec alloziiert werden. Da der malloc Befehl zum alloziieren verwendet wird muss der Speicher sp"ater wieder mit einem Aufruf von free freigegeben werden. Hierzu dient die Funktion delete im Standardbinding c.
Die letzten 4 Funktionen dienen zum Zugriff auf die Membervariablen der Struktur.\\
\\
Es ist anzumerken, dass auch diese automatisch erstellten Funktionen die Filter passieren m"ussen. So kann unter anderem durch folgenden Eintrag verhindert werden, dass man schreibend auf einen Vektor zugreift:

\begin{verbatim}
<filter>
  <ignore>vec_set##</ignore>
  <accept>##</accept>
</filter>
\end{verbatim}

%%%%%%%%%%%%%%%%%%%%%%%%%%%%%%%%%%%%%%%%%%%%%%%%%%%%%%%%%%%%%%%%%%%%%%%
%%%%%%%%%%%%%%%%%%%%%%%%%%%%%%%%%%%%%%%%%%%%%%%%%%%%%%%%%%%%%%%%%%%%%%%
\section{Standardbinding}

Es wird ein kleines Binding Namens c bereitgestellt, welches Grundfunktionen
der Programmiersprache C auf Alice-Seite bereitstellt. Beispielsweise das
Casting von Zeigertypen und "ahnliches. Hier eine Liste der in diesem Binding enthaltenen
Funktionen:

\begin{verbatim}
val new : int -> 'a pointer
val delete : 'a pointer -> unit
val pointer : 'a pointer -> 'b pointer pointer
val unref : 'a pointer pointer -> 'b pointer
val cast : 'a pointer -> 'b pointer

val sizeof_char : unit -> int
val cast_char : 'a pointer -> c_char pointer
val pointer_char : char -> c_char pointer
val unref_char : c_char pointer -> char

val sizeof_short : unit -> int
val cast_short : 'a pointer -> c_short pointer
val pointer_short : int -> c_short pointer
val unref_short : c_short pointer -> int

val sizeof_int : unit -> int
val cast_int : 'a pointer -> c_int pointer
val pointer_int : int -> c_int pointer
val unref_int : c_int pointer -> int

val sizeof_long : unit -> int
val cast_long : 'a pointer -> c_long pointer
val pointer_long : int -> c_long pointer
val unref_long : c_long pointer -> int

val sizeof_float : unit -> int
val cast_float : 'a pointer -> c_float pointer
val pointer_float : real -> c_float pointer
val unref_float : c_float pointer -> real

val sizeof_double : unit -> int
val cast_double : 'a pointer -> c_double pointer
val pointer_double : real -> c_double pointer
val unref_double : c_double pointer -> real
\end{verbatim}

Mit new kann Speicher alloziiert werden und mit delete wieder freigegeben werden.
Die Funktion pointer wandelt einen Zeiger in einen Zeiger auf einen Zeiger um. Dabei wird 
Speicher alloziiert, der mit delete freigegeben werden muss. Unref stellt die umgekehrte Funktion dar, Speicher wird jedoch \emph{nicht} freigegeben. Die Funktion cast stellt einen Cast von einem beliebigen Zeiger in einen anderen dar.\\
F"ur die wichtigsten C-Basistypen sind Funktionen zum ermitteln der Gr"o"se dieses Typs in Bytes, zum typsicheren Casten von Zeigern, sowie zum Referenzieren und Dereferenzieren vorhanden.

%%%%%%%%%%%%%%%%%%%%%%%%%%%%%%%%%%%%%%%%%%%%%%%%%%%%%%%%%%%%%%%%%%%%%%%
%%%%%%%%%%%%%%%%%%%%%%%%%%%%%%%%%%%%%%%%%%%%%%%%%%%%%%%%%%%%%%%%%%%%%%%
\section{Beispielbinding}

Hier ein kleines Beispiel: \\

Datei util.h:\\
--------------------------------------------------------------------- 

\begin{verbatim}
#include <stdio.h>
#include <gtk/gtk.h>

void printhello();

char add_char(char a, char b);
short add_short(short a, short b);
long add_long(long a, long b);
long long add_longlong(long long a, long long b);
int add_int(int a, int b);

float add_float(float a, float b);
double add_double(double a, double b);
long double add_longdouble(long double a, long double b);

int add_array(char a[4]);

typedef enum { const_a,const_b,const_c,const_d } myenum;


struct vec {
  int x; int y;
  myenum a;
};

struct vec* add_vec(struct vec* x, struct vec* y);

myenum dummy(myenum p);

void mymemcpy(void* dst, const void* src, int length);
void mymemset(void* dst, int length, char chr);

void inc(int* i,...);

int list_add(GList* lst);

typedef void (*voidtovoid)(void);
voidtovoid get_printhello();
void call_fun(voidtovoid f);

\end{verbatim}

Datei util.cc: \\
--------------------------------------------------------------------- 

\begin{verbatim}
#include <stdio.h>
#include <glib.h>

#include "util.h" 

// Einfaches Printhello
void printhello() {
  printf("Hello World!\n");
}

// Addition auf Integern
char add_char(char a, char b) { return a + b; }
short add_short(short a, short b) { return a + b; }
long add_long(long a, long b) { return a + b; }
long long add_longlong(long long a, long long b) { return a + b; }
int add_int(int a, int b) { return a + b; }

// Addition auf Flie�kommazahlen
float add_float(float a, float b) { return a + b; }
double add_double(double a, double b) { return a + b; }
long double add_longdouble(long double a, long double b) { return a + b; }

// Addiert die Werte eines Arrays
int add_array(char a[4]) {
  int r = 0;
  for (int i=0; i<4; i++)
    r+=a[i];
  return r;
}

// Addiert zwei Vektoren
vec* add_vec(vec* a, vec* b) {
  vec* v = new vec;
  v->x = a->x + b->x;
  v->y = a->y + b->y;
  return v;
}

// mymemcpy
void mymemcpy(void* dst, const void* src, int length) {
  char* cdst = (char*) dst;
  char* csrc = (char*) src;

  while (length--) {
      cdst[0] = csrc[0];
      cdst++; csrc++;
    }
}

// mymemset
void mymemset(void* dst, int length, char chr) {
  char* cdst = (char*) dst;
  while (length--) {
      cdst[0] = chr;
      cdst++; 
    }
}

// Inc
void inc(int* p,...) {  p[0]++; }

// GList
int list_add(GList* lst) {
  int s = 0;
  GList* frst = g_list_first(lst);
  for (int i=0; i<g_list_length(frst); i++) {
    int* data = (int*)g_list_nth_data(frst,i);
    s += *data;
  }
  return s;
}

// dummy
myenum dummy(myenum p) {
  printf("Value: %i\n",p);
  return (myenum)((int)p+1);
}

void call_fun(void (*f)(void)) {  f(); }
voidtovoid get_printhello() { return printhello; }

\end{verbatim}

Konfigurationsdatei util-config.xml: \\
--------------------------------------------------------------------- 

\begin{verbatim}
<xml>

<!-- Filter die auf alle automatisch generierten 
     Funktionen und Typen angewandt werden -->

<filter>

  <accept>pointer</accept>
  <accept>c_##</accept>

  <accept>new</accept>
  <accept>delete</accept>
  <accept>cfun</accept>
  <accept>pointer#x#</accept>
  <accept>unref#x#</accept>
  <accept>sizeof#x#</accept>
  <accept>cast#x#</accept>

  <accept>printhello</accept>
  <accept>add_#x#</accept>
  <accept>vec##</accept>
  <accept>inc</accept>
  <accept>list_#x#</accept>
  <accept>const_#x#</accept>
  <accept>dummy</accept>
  <accept>call_fun</accept>
  <accept>voidtovoid</accept>
  <accept>get_printhello</accept>

  <rename>
    <from>my#x#</from>
    <to>#x#</to>
  </rename>

</filter>


<filter>
  <rename><from>_##</from><to>##</to></rename>
  <rename><from>__##</from><to>##</to></rename>
  <rename><from>___##</from><to>##</to></rename>
  <accept>##</accept>
</filter>

<!-- Benutzerdefinierte Funktionen -->

<fun>
   <name>mycast</name>
   <type>'a -> 'b</type>
   <export>INIT_STRUCTURE(record, "util", "mycast", mycast, 1);</export>
   <body>
       DEFINE1(mycast) {
         RETURN(x0);
       } END
   </body>
</fun>


<!-- Spezielle Typbehandlung -->

<type>
  <ctype>GList*</ctype>
  <alicetype>'a pointer list</alicetype>
  <toword>GTK_LIST_TO_WORD</toword>
  <fromword>DECLARE_GTK_LIST</fromword>
</type>

</xml>
\end{verbatim}

Erstellte Signatur util-sig.aml: \\
--------------------------------------------------------------------- 

\begin{verbatim}
import structure types from "types"
signature util =
sig
   type 'a pointer = 'a types.pointer
   type  c_fun =  types.c_fun
   type  c_char =  types.c_char
   type  c_short =  types.c_short
   type  c_int =  types.c_int
   type  c_long =  types.c_long
   type  c_longlong =  types.c_longlong
   type  c_float =  types.c_float
   type  c_double =  types.c_double
   type  c_longdouble =  types.c_longdouble

   val printhello : unit -> unit
   val add_char : char * char -> char
   val add_short : int * int -> int
   val add_long : int * int -> int
   val add_longlong : int * int -> int
   val add_int : int * int -> int
   val add_float : real * real -> real
   val add_double : real * real -> real
   val add_longdouble : real * real -> real
   val add_array : char Array.array -> int

   datatype enum = const_a | const_b | const_c | const_d

   type  vec
   val cast_vec : 'a pointer -> vec pointer
   val sizeof_vec : unit -> int
   val vec_new : unit -> vec pointer
   val vec_get_x : vec pointer -> int
   val vec_set_x : vec pointer * int -> unit
   val vec_get_y : vec pointer -> int
   val vec_set_y : vec pointer * int -> unit
   val vec_get_a : vec pointer -> enum
   val vec_set_a : vec pointer * enum -> unit

   val add_vec : vec pointer * vec pointer -> vec pointer
   val dummy : enum -> enum
   val memcpy : 'a pointer * 'b pointer * int -> unit
   val memset : 'a pointer * int * char -> unit
   val inc : c_int pointer -> unit

   type voidtovoid
   val get_printhello : unit -> voidtovoid
   val call_fun : voidtovoid -> unit
   val mycast : 'a -> 'b
end
\end{verbatim}

Erstellte Datei util.asig: \\
--------------------------------------------------------------------- 

\begin{verbatim}

import signature util from "util-sig"
signature util =
sig
        structure util: util
end
\end{verbatim}

Testprogramm f"ur das Binding:\\
--------------------------------------------------------------------- 

\begin{verbatim}
import structure c from "c"
import structure util from "util"

open c
open util

val _ = printhello();

val v1 = vec_new();
val _ = memset(v1,sizeof_vec(),#" ");

val v2 = vec_new();
val _ = memset(v2,sizeof_vec(),#" ");

val _ = vec_set_x(v1,1);
val _ = vec_set_y(v1,2);
val _ = memcpy(v2,v1,sizeof_vec());

val vsum = add_vec(v1,v2);

val _ = print ("vsum = ("^Int.toString(vec_get_x vsum)^","^
                          Int.toString(vec_get_y vsum)^")\n");


val v3 = cast_vec(new(sizeof_vec()));
val _ = memcpy(v3,v2,sizeof_vec());

val _ = print ("v3 = ("^Int.toString(vec_get_x v3)^","^
                        Int.toString(vec_get_y v3)^")\n");

val v3x = unref_int (cast_int v3)
val _ = print ("v3.x = "^Int.toString(v3x)^"\n")

val arr = Array.array(4,#"a")
val sum1 = add_array(arr);
val _ = print ("Summe Array: "^Int.toString(sum1)^"\n");

val _ = let val lst = map pointer_int [1,2,3,4,5]
    val sum2 = list_add lst
    val _ = map delete lst
in
    print ("Summe Liste: "^Int.toString(sum2)^"\n")
end

val sum3 = (mycast const_a) + (mycast const_b) + (mycast const_c) + (mycast const_d);
val _ = print ("Summe a+b+c+d: "^Int.toString(sum3)^"\n")

val pi = pointer_int 4
val _ = inc pi
val i = unref_int pi
val _ = delete pi
val _ = print ("Nach Inkrement: "^Int.toString(i)^"\n")

val _ = dummy const_a

val _ = OS.Process.exit OS.Process.success
\end{verbatim}

Das lauff"ahige Beispielbinding ist im Verzeichnis sample zu finden. Es existiert im Verzeichnis gtk weiterhin ein Binding der GTK-Bibliothek mit lauff"ahigen Demoprogrammen (hello, scramble).


%%%%%%%%%%%%%%%%%%%%%%%%%%%%%%%%%%%%%%%%%%%%%%%%%%%%%%%%%%%%%%%%%%%%%%%
%%%%%%%%%%%%%%%%%%%%%%%%%%%%%%%%%%%%%%%%%%%%%%%%%%%%%%%%%%%%%%%%%%%%%%%
\section{Nicht unterst"utzt}

Nicht unterst"utzt sind folgende Features:

\begin{itemize}
\item Funktionen mit variabler Parameteranzahl, wie z.B.\\ void printf(char* buf, ...). Momentan werden die variablen Parameter einfach ignoriert, also eine Funktion folgenden Typs im Binding erstellt \\ c\_char pointer $\rightarrow$ unit.
\item Es k"onnen keine Alice-Funktionen an gebundene Funktionen "ubergeben werden. C-Code kann also keinen Alice-Code aufrufen. 
\end{itemize}


%%%%%%%%%%%%%%%%%%%%%%%%%%%%%%%%%%%%%%%%%%%%%%%%%%%%%%%%%%%%%%%%%%%%%%
%%%%%%%%%%%%%%%%%%%%%%%%%%%%%%%%%%%%%%%%%%%%%%%%%%%%%%%%%%%%%%%%%%%%%%%

\bibliographystyle{unsrt}
\nocite*
\bibliography{quellen}

\end{document}
